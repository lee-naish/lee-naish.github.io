\begin{abstract}
% \begin{center}
% \textbf{Abstract}
% \end{center}
Declarative (or algorithmic) debugging is a promising technique for
debugging lazy functional programs.  This paper addresses two important
reasons why it is not more widespread: the difficulty of
writing a declarative debugger for a lazy functional language in the
language itself and the efficiency of the debugger.   Using the source
language to implement the debugger is desirable for portability and we
discuss the reasons why this is very difficult to do.
We propose a system in which nearly all
the code is in the source language but there is one function which must
be written at lower level.  We also show how this function can be the
key to significantly improving the efficiency of declarative debuggers for
such languages.

\noindent
\textbf{Keywords}: algorithmic debugging, referential transparency,
i/o, monads, types, trace, transformation
\end{abstract}
